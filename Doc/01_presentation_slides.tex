% Options for packages loaded elsewhere
\PassOptionsToPackage{unicode}{hyperref}
\PassOptionsToPackage{hyphens}{url}
%
\documentclass[
  ignorenonframetext,
]{beamer}
\usepackage{pgfpages}
\setbeamertemplate{caption}[numbered]
\setbeamertemplate{caption label separator}{: }
\setbeamercolor{caption name}{fg=normal text.fg}
\beamertemplatenavigationsymbolsempty
% Prevent slide breaks in the middle of a paragraph
\widowpenalties 1 10000
\raggedbottom
\setbeamertemplate{part page}{
  \centering
  \begin{beamercolorbox}[sep=16pt,center]{part title}
    \usebeamerfont{part title}\insertpart\par
  \end{beamercolorbox}
}
\setbeamertemplate{section page}{
  \centering
  \begin{beamercolorbox}[sep=12pt,center]{part title}
    \usebeamerfont{section title}\insertsection\par
  \end{beamercolorbox}
}
\setbeamertemplate{subsection page}{
  \centering
  \begin{beamercolorbox}[sep=8pt,center]{part title}
    \usebeamerfont{subsection title}\insertsubsection\par
  \end{beamercolorbox}
}
\AtBeginPart{
  \frame{\partpage}
}
\AtBeginSection{
  \ifbibliography
  \else
    \frame{\sectionpage}
  \fi
}
\AtBeginSubsection{
  \frame{\subsectionpage}
}
\usepackage{lmodern}
\usepackage{amssymb,amsmath}
\usepackage{ifxetex,ifluatex}
\ifnum 0\ifxetex 1\fi\ifluatex 1\fi=0 % if pdftex
  \usepackage[T1]{fontenc}
  \usepackage[utf8]{inputenc}
  \usepackage{textcomp} % provide euro and other symbols
\else % if luatex or xetex
  \usepackage{unicode-math}
  \defaultfontfeatures{Scale=MatchLowercase}
  \defaultfontfeatures[\rmfamily]{Ligatures=TeX,Scale=1}
\fi
% Use upquote if available, for straight quotes in verbatim environments
\IfFileExists{upquote.sty}{\usepackage{upquote}}{}
\IfFileExists{microtype.sty}{% use microtype if available
  \usepackage[]{microtype}
  \UseMicrotypeSet[protrusion]{basicmath} % disable protrusion for tt fonts
}{}
\makeatletter
\@ifundefined{KOMAClassName}{% if non-KOMA class
  \IfFileExists{parskip.sty}{%
    \usepackage{parskip}
  }{% else
    \setlength{\parindent}{0pt}
    \setlength{\parskip}{6pt plus 2pt minus 1pt}}
}{% if KOMA class
  \KOMAoptions{parskip=half}}
\makeatother
\usepackage{xcolor}
\IfFileExists{xurl.sty}{\usepackage{xurl}}{} % add URL line breaks if available
\IfFileExists{bookmark.sty}{\usepackage{bookmark}}{\usepackage{hyperref}}
\hypersetup{
  pdftitle={R for Bio Data Science},
  pdfauthor={Group 01: Sara, Annie, Wellace and Bokai},
  hidelinks,
  pdfcreator={LaTeX via pandoc}}
\urlstyle{same} % disable monospaced font for URLs
\newif\ifbibliography
\usepackage{color}
\usepackage{fancyvrb}
\newcommand{\VerbBar}{|}
\newcommand{\VERB}{\Verb[commandchars=\\\{\}]}
\DefineVerbatimEnvironment{Highlighting}{Verbatim}{commandchars=\\\{\}}
% Add ',fontsize=\small' for more characters per line
\usepackage{framed}
\definecolor{shadecolor}{RGB}{248,248,248}
\newenvironment{Shaded}{\begin{snugshade}}{\end{snugshade}}
\newcommand{\AlertTok}[1]{\textcolor[rgb]{0.94,0.16,0.16}{#1}}
\newcommand{\AnnotationTok}[1]{\textcolor[rgb]{0.56,0.35,0.01}{\textbf{\textit{#1}}}}
\newcommand{\AttributeTok}[1]{\textcolor[rgb]{0.77,0.63,0.00}{#1}}
\newcommand{\BaseNTok}[1]{\textcolor[rgb]{0.00,0.00,0.81}{#1}}
\newcommand{\BuiltInTok}[1]{#1}
\newcommand{\CharTok}[1]{\textcolor[rgb]{0.31,0.60,0.02}{#1}}
\newcommand{\CommentTok}[1]{\textcolor[rgb]{0.56,0.35,0.01}{\textit{#1}}}
\newcommand{\CommentVarTok}[1]{\textcolor[rgb]{0.56,0.35,0.01}{\textbf{\textit{#1}}}}
\newcommand{\ConstantTok}[1]{\textcolor[rgb]{0.00,0.00,0.00}{#1}}
\newcommand{\ControlFlowTok}[1]{\textcolor[rgb]{0.13,0.29,0.53}{\textbf{#1}}}
\newcommand{\DataTypeTok}[1]{\textcolor[rgb]{0.13,0.29,0.53}{#1}}
\newcommand{\DecValTok}[1]{\textcolor[rgb]{0.00,0.00,0.81}{#1}}
\newcommand{\DocumentationTok}[1]{\textcolor[rgb]{0.56,0.35,0.01}{\textbf{\textit{#1}}}}
\newcommand{\ErrorTok}[1]{\textcolor[rgb]{0.64,0.00,0.00}{\textbf{#1}}}
\newcommand{\ExtensionTok}[1]{#1}
\newcommand{\FloatTok}[1]{\textcolor[rgb]{0.00,0.00,0.81}{#1}}
\newcommand{\FunctionTok}[1]{\textcolor[rgb]{0.00,0.00,0.00}{#1}}
\newcommand{\ImportTok}[1]{#1}
\newcommand{\InformationTok}[1]{\textcolor[rgb]{0.56,0.35,0.01}{\textbf{\textit{#1}}}}
\newcommand{\KeywordTok}[1]{\textcolor[rgb]{0.13,0.29,0.53}{\textbf{#1}}}
\newcommand{\NormalTok}[1]{#1}
\newcommand{\OperatorTok}[1]{\textcolor[rgb]{0.81,0.36,0.00}{\textbf{#1}}}
\newcommand{\OtherTok}[1]{\textcolor[rgb]{0.56,0.35,0.01}{#1}}
\newcommand{\PreprocessorTok}[1]{\textcolor[rgb]{0.56,0.35,0.01}{\textit{#1}}}
\newcommand{\RegionMarkerTok}[1]{#1}
\newcommand{\SpecialCharTok}[1]{\textcolor[rgb]{0.00,0.00,0.00}{#1}}
\newcommand{\SpecialStringTok}[1]{\textcolor[rgb]{0.31,0.60,0.02}{#1}}
\newcommand{\StringTok}[1]{\textcolor[rgb]{0.31,0.60,0.02}{#1}}
\newcommand{\VariableTok}[1]{\textcolor[rgb]{0.00,0.00,0.00}{#1}}
\newcommand{\VerbatimStringTok}[1]{\textcolor[rgb]{0.31,0.60,0.02}{#1}}
\newcommand{\WarningTok}[1]{\textcolor[rgb]{0.56,0.35,0.01}{\textbf{\textit{#1}}}}
\usepackage{longtable,booktabs}
\usepackage{caption}
% Make caption package work with longtable
\makeatletter
\def\fnum@table{\tablename~\thetable}
\makeatother
\setlength{\emergencystretch}{3em} % prevent overfull lines
\providecommand{\tightlist}{%
  \setlength{\itemsep}{0pt}\setlength{\parskip}{0pt}}
\setcounter{secnumdepth}{-\maxdimen} % remove section numbering

\title{R for Bio Data Science}
\author{Group 01: Sara, Annie, Wellace and Bokai}
\date{}

\begin{document}
\frame{\titlepage}

\begin{frame}{Introduction}
\protect\hypertarget{introduction}{}

R Markdown is built into RStudio and allows you to create documents like
HTML, PDF, and Word documents from R. With R Markdown, you can embed R
code into your documents.

\begin{block}{Why use R Markdown?}

\begin{itemize}
\tightlist
\item
  Turn work in R into more accessible formats
\item
  Incorporate R code and R plots into documents
\item
  R Markdown documents are reproducible -- the source code gets rerun
  every time a document is generated, so if data change or source code
  changes, the output in the document will change with it.
\end{itemize}

\end{block}

\end{frame}

\begin{frame}{Getting Started}
\protect\hypertarget{getting-started}{}

\begin{itemize}
\tightlist
\item
  Create a new R Markdown file in RStudio by going to\\
  File \textgreater{} New File \textgreater{} R Markdown\ldots{}
\item
  Click the ``presentation'' tab
\item
  Enter a title, author, and select what kind of slideshow you
  ultimately want (this can all be changed later)
\end{itemize}

\end{frame}

\begin{frame}[fragile]{Getting Started}
\protect\hypertarget{getting-started-1}{}

The beginning of an R Markdown file looks like this: \texttt{-\/-\/-}\\
\texttt{title:\ "Air\ Quality"}~\\
\texttt{author:\ "JHU"}~\\
\texttt{date:\ "May\ 17,\ 2016"}~\\
\texttt{output:\ html\_document}~\\
\texttt{-\/-\/-} The new document you've created will contain example
text and code below this -- delete it for a fresh start.

\end{frame}

\begin{frame}[fragile]{Making Your First Slide}
\protect\hypertarget{making-your-first-slide}{}

\begin{itemize}
\tightlist
\item
  Title your first slide using two \# signs:
  \texttt{\#\#\ Insert\ Title\ Here}
\item
  To make a slide without a title, use three asterisks: \texttt{***}
\item
  You can add subheadings with more \# signs:
  \texttt{\#\#\#\ Subheading} or \texttt{\#\#\#\#\ Smaller\ Subheading}
\item
  To add a new slide, just add another Title:
  \texttt{\#\#\ New\ Slide\ Title}
\end{itemize}

\end{frame}

\begin{frame}[fragile]{Adding Text}
\protect\hypertarget{adding-text}{}

\begin{itemize}
\tightlist
\item
  Add bullet points to a slide using a hyphen followed by a space:
  \texttt{-\ bullet\ point}
\item
  Add sub-points using four spaces and a plus sign:
  ~~~~\texttt{+\ sub-point}
\item
  Add an ordered list by typing the number/letter:
  \texttt{1.\ first\ point} ~~~~\texttt{a)\ sub-sub-point}
\item
  Add bullet points that appear one by one (on click) with:
  \texttt{\textgreater{}-\ iterated\ bullet\ point}
\end{itemize}

\end{frame}

\begin{frame}[fragile]{Formatting Text}
\protect\hypertarget{formatting-text}{}

\begin{longtable}[]{@{}cc@{}}
\toprule
Text & Code in R Markdown\tabularnewline
\midrule
\endhead
plain text & \texttt{plain\ text}\tabularnewline
\emph{italics} & \texttt{*italics*}\tabularnewline
\textbf{bold} & \texttt{**bold**}\tabularnewline
\href{http://www.jhsph.edu}{link} &
\texttt{{[}link{]}(http://www.jhsph.edu)}\tabularnewline
\texttt{verbatim\ code} & \texttt{}code here\texttt{}\tabularnewline
\bottomrule
\end{longtable}

\end{frame}

\begin{frame}[fragile]{Embedding R Code}
\protect\hypertarget{embedding-r-code}{}

This is a chunk of R code in R Markdown: ```\{r\}
\texttt{head(airquality)} ```\\
The code gets run, and both the input and output are displayed.

\begin{Shaded}
\begin{Highlighting}[]
\KeywordTok{head}\NormalTok{(airquality)}
\end{Highlighting}
\end{Shaded}

\begin{verbatim}
##   Ozone Solar.R Wind Temp Month Day
## 1    41     190  7.4   67     5   1
## 2    36     118  8.0   72     5   2
## 3    12     149 12.6   74     5   3
## 4    18     313 11.5   62     5   4
## 5    NA      NA 14.3   56     5   5
## 6    28      NA 14.9   66     5   6
\end{verbatim}

\end{frame}

\begin{frame}[fragile]{Embedding R Code}
\protect\hypertarget{embedding-r-code-1}{}

To hide the input code, use \texttt{echo=FALSE}. ```\{r, echo=FALSE\}
\texttt{head(airquality)} ```

\begin{verbatim}
##   Ozone Solar.R Wind Temp Month Day
## 1    41     190  7.4   67     5   1
## 2    36     118  8.0   72     5   2
## 3    12     149 12.6   74     5   3
## 4    18     313 11.5   62     5   4
## 5    NA      NA 14.3   56     5   5
## 6    28      NA 14.9   66     5   6
\end{verbatim}

This can be useful for showing plots.

\end{frame}

\begin{frame}[fragile]{Embedding R Code}
\protect\hypertarget{embedding-r-code-2}{}

To show the input code only, use \texttt{eval=FALSE}. ```\{r,
eval=FALSE\} \texttt{head(airquality)} ```

\begin{Shaded}
\begin{Highlighting}[]
\KeywordTok{head}\NormalTok{(airquality)}
\end{Highlighting}
\end{Shaded}

\end{frame}

\begin{frame}[fragile]{Embedding R Code}
\protect\hypertarget{embedding-r-code-3}{}

To run the code without showing input or output, use
\texttt{include=FALSE}. ```\{r, include=FALSE\}
\texttt{library(ggplot2)} ```

\end{frame}

\begin{frame}[fragile]{Generating Slideshows}
\protect\hypertarget{generating-slideshows}{}

\begin{itemize}
\tightlist
\item
  Click the \textbf{Knit} button at the top of the R Markdown document
  to generate your new document.

  \begin{itemize}
  \tightlist
  \item
    You may be asked to install required packages if you don't already
    have them installed -- hit ``Yes'' and RStudio will install them for
    you
  \end{itemize}
\item
  You can change the type of document generated by changing the
  \texttt{output} line in the header, or by selecting an output from the
  \textbf{Knit} button's pull-down menu.
\end{itemize}

\end{frame}

\begin{frame}[fragile]{Generating Slideshows}
\protect\hypertarget{generating-slideshows-1}{}

\begin{itemize}
\tightlist
\item
  HTML: two options with different looks

  \begin{itemize}
  \tightlist
  \item
    \texttt{output:\ ioslides\_presentation}
  \item
    \texttt{output:\ slidy\_presentation}
  \end{itemize}
\item
  PDF: \texttt{output:\ beamer\_presentation}\\
\item
  Note: You can specify multiple outputs at the beginning of the R
  Markdown file if you will need to generate multiple filetypes.
\end{itemize}

\end{frame}

\begin{frame}{PDFs and LaTeX}
\protect\hypertarget{pdfs-and-latex}{}

\begin{itemize}
\tightlist
\item
  To \textbf{knit} a PDF slideshow, you will need to install
  \textbf{LaTeX} on your computer
\item
  LaTeX is a typesetting system that is needed to convert R Markdown
  into formatted text for PDFs
\end{itemize}

\begin{block}{Downloading and Installing LaTeX}

\begin{itemize}
\tightlist
\item
  \emph{LaTeX} is free
\item
  LaTeX takes up a lot of space (almost \textasciitilde2.6 GB download
  and takes up \textasciitilde5 GB when installed)
\item
  Visit \url{https://www.tug.org/begin.html} to download LaTeX for your
  operating system
\item
  Depending on your internet connection, it may take a while to download
  due to its size
\end{itemize}

\end{block}

\end{frame}

\begin{frame}{Conclusion}
\protect\hypertarget{conclusion}{}

For more information about R Markdown visit
\url{http://rmarkdown.rstudio.com/}

\end{frame}

\end{document}
